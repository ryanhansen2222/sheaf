\documentclass[10.5pt]{article}
\usepackage{uspstyle}
\usepackage{graphicx}
\usepackage{mathrsfs}
\usepackage{amsfonts}
\begin{document}





% The content of your research proposal, including text, figures and images, 
% should not exceed 5 pages, typed, double-spaced (additional pages may be 
% included for your works-cited/references).
\maketitle

\section{Introduction}
% Provide a statement of the objective(s) and the anticipated significance of the 
% work to your field of study. What problems will be investigated? What hypothesis 
% will be tested? We suggest that the introduction begin with a brief description 
% of the project in general terms before the more technical aspects of the project 
% are discussed. 
%\todo{put some sort of high-level introduction here.}
%here is how you cite a figure (be sure to updated the ref here ... and the 
%label inside the figure

One of the major focuses of science, especially physical sciences, is to 
predict outcomes of events. A common technique to develop predictions is to fit 
data to mathematical models. However, depending on the complexity of the 
mathematics, or the 
phenomena being modeled, how well a given model fits the data is sometimes 
difficult to gauge. The 
primary example, and the focus of this research project, is metabolomics - the 
model constructed to predict metabolic responses to stimuli. There are two main 
issues associated with metabolomics. First, data is usually noisy. Samples are 
often taken from a wide variety of sources, including differing tissues, or 
even organisms. One of our major goals for this project is to develop a way to 
measure model fitness with high resilience to noise in data. Sheaves are 
particularly good at adapting to noise \cite{robinson2017sheaves}. Second, 
metabolism is generally complicated. Dozens of reactions occur simultaneously, 
some of which are more under the hood than others. In particular, reactions are 
sometimes not considered during analysis because including them leads to 
uncountably infinite solutions, reduced computation efficiency, or even require 
unmeasurable data. For this reason, sheaves stick out as a useful analysis 
tool. The second focal point of this project 
is to not only find if there is likely a poor or missing reaction from our 
model, but also to help determine what this reaction deals with. For example, 
if we try to use a simplified model for glycolysis, the metabolite data will 
likely vary slightly more than what we expect from a more rigorous model. Using 
sheaves, we will not only indicate our simplified model is probably missing 
some reactions, but should also point to certain sets of metabolites that are 
likely involved in the missing reactions. Finally, we note this technique can 
be abstracted from exclusively looking at metabolomics. 

%Figure~\ref{fig:TODO-name}

%TO cite something, use~\cite{fasy2014statistical}.

% \begin{figure}
%  \centering
%  \includegraphics[height=1.25in]{TODO:relative-path}
%  \caption{TODO.}\label{fig:TODO-name}
% \end{figure}

\section{Background}
% Provide a brief review of the work that has been done in the project 
% area together with complete references in appropriate professional style. This 
% section should also include any personal information about you that would 
% indicate to the reviewers your qualifications for successfully completing this 
% project, including a statement of how the project will contribute to your 
% academic and career goals.
%\todo{discuss related work, and why you are qualified to work on this.  Perhaps 
%make subsections.}

\subsection{Personal Value}
There are a number of reasons why pursuing computational topology and geometry 
research at MSU is valuable to me. First and foremost, I enjoy researching it. 
I have always been interested in exploring math related problems and developing 
solutions to them. Second, it develops my personal skills. Through close 
guidance by the experienced members of my group, I have grown my technical 
skills, communication skills, and familiarized myself with the way new ideas 
are developed and shared with the rest of the world. Third, it is awesome to 
explore open questions.
\subsection{Qualifications}
I am a fourth year student majoring in Physics and Computer Science and 
minoring in Math. This means I have a fair amount of experience in coding and 
the relevant mathematics, both of which are major components of this project. 
Furthermore, I have taken or am taking Linear Algebra, General Topology, 
Computational Topology, and the standard track for Computer Science students. 
Aside from courses, I have been an active member in this research group since 
2018, hold an officer position in Math Club, and regularly participate in 
various academic events (ie Putnam, COMAP).


\section{Methods}
% Provide a detailed description of the research methods that you will 
% use in the project. This should include a justification for the specific 
% approach that you will use. For example, how do the methods answer the questions 
% that have been posed, test the hypothesis, or lead to the desired goal?
% Timeline: Provide dates for the initiation and completion of each phase of the 
% project. Attempt to lay out a reasonable schedule taking into consideration all 
% phases of the research and final deliverables.
As commonly defined, let the sheaf $\mathscr{F}$ be a
contravariant functor. Our goal is to apply the tools of a sheaf to metabolic 
data to handle the high-noise and high-dimensional issues it brings. As it is a 
functor, we must first define the category in the domain of the sheaf, the
category of reactions, $\mathcal{X}$. Then, we describe a mapping from the
objects and morphisms of $\mathcal{X}$ into the category in the image of the
sheaf functor, which we denote by $E$, a category of real Euclidean spaces. 

\subsection{The Category of Reactions}
As mentioned earlier, metabolic reactions are currently modeled as a set 
$\mathcal{R}$ of stoichiometric equations. Each of these equations contains a 
set of reactants, $M_0$, a set of products, $M_1$. Collectively, these are the 
\textit{objects} of this category. Additionally, our model quantifies how the 
reactants change to products. These are the \textit{morphisms}. Each reaction 
occurs at some rate. This means over some time interval, each reaction occurs 
some number of times, $r_m$. Given constants to describe the morphisms, 
$c_i,c_j \in \mathbb{Z}^+$, these reactions are of the form \begin{equation}
r_m\sum_{m_i \in M_0} c_i m_i \rightarrow r_m\sum_{m_j \in M_1} c_j m_j .
\end{equation}  
Together, the objects (reactants and products) and morphisms (rates and constants) completely describe the category of reactions.
\subsection{$\mathscr{F}$ and its image}
$\mathscr{F}$ is a tool that transforms the initial model to something more useful to analyze. It maintains the same core structure, though, by mapping objects to objects and morphisms to morphisms. 
We want to connect our model to measurable data. This data is recorded by 
measured quantities of our reactant and product parameters. Given $n$ unique 
parameters (reactants and/or products), our data will subsequently lie in 
$\mathbb{R}^n$. Since we measure them one at a time, it makes sense, then, to 
define the mapping in a way to isolate each parameter. We do this by utilizing 
the reaction rate, $r_i$. We know if a reaction happens $r_i$ times, then $r_i$ 
sets of reactants will be lost and $r_i$ sets of products will be gained. If we 
follow this process throughout the entire set of reactions and arrange them 
based on parameter, we subsequently have a mapping from the category of 
reactions to $\mathbb{R}^n$, or $E$, the category of real Euclidean spaces, 
where our data can be realized. This is how the sheaf maps the objects of the 
category of reactions. When we mandate that we maintain the same structure as 
the initial reaction morphisms, it turns out that the mappings in the image of 
$\mathscr{F}$ are simply matrix multiplication of our input set of reactions.
\subsection{Consistency}
Recall our main goal is to verify if a model is consistent or not. Using the 
aforementioned process, we can obtain the model's space of all permitted data. 
If our data's shape is significantly different than what we expect, we want to 
be able to conclude something about our model is not correct. Fortunately, 
there is already a sheaf tool defined by Michael Robinson called the 
"Consistency Radius" to quantify the error propagated through our model 
\cite{robinson2017sheaves}.

\subsection{Example}
Below is an example of our sheaf applied to a simple system of equations. With 
the sheaf framework, we can easily determine which portions of our model are 
consistent with observation, and which are not. The \textit{Assignment} 
indicates what the data measured for a given metabolite. The 
\textit{Restriction Value} refers to what our model expects the data value to 
be, given its initialization with the above data and reaction equations. Since 
two arrows point to $r_1$, it has two restriction values. If we measure $\Delta 
B = 2, r_1=2$, we know for certain there is a deviation from our measured value 
and expected value, as shown in the figure below. We call the consistency 
radius the maximum of these deviations, so a larger consistency radius is an 
indication of a worse model fit. In this scenario, the consistency radius would 
be 4. Additionally, we can say since the error came from metabolite $r_1$ and 
$B$, that there is likely something missing or incorrect coming from a reaction 
related to those components.
\begin{figure}[h]
	\caption{\textit{Our sheaf applied to the reactions $A + 2B\rightarrow C$ at rate r1 and $C \rightarrow A$ at rate r2}}
	
	\centering
	\includegraphics[width=0.7\textwidth, 
	height=0.25\textheight]{examplemapping}
\end{figure}
\subsection{USP Goals}
\begin{enumerate}
	\item Experiment:
	I have recently been trained to use the HPC Hyalite Cluster for the purpose 
	of running a data simulation or, if things work out, testing on real data. 
	\item Simulation Analysis: After simulating, we want to draw conclusions 
	about the strengths and weaknesses of using sheaves as a measure of model 
	fitness
	\item Paper: One of my biggest goals this USP is to be published by the 
	end. 

\end{enumerate}
\subsection{Experiment}
\begin{enumerate}
	\item Generate sets of reaction equations. Generate metabolite data from 
	reaction equations using the a normal distribution and 5\% standard 
	deviation
	\item Test each data set on each set of reaction equations. We expect low 
	consistency radius if the data is generated from that model, and higher 
	consistency radius is the data is generated from some other model.
	\item Analyze (Come to conclusions about strengths and weaknesses of sheaf 
	application). Graph consistency radius vs changes in topologies.
\end{enumerate}
\subsection{Potential Future Related Problems}
Two interesting questions arise from this project. First, it is questionable 
when a model is "good enough" to be accepted. Incorporating statistical 
confidence intervals is a tried and true method for many situations, and 
adapting it to sheaf model fitness is an interesting route. Second, we have 
developed this idea of quantifying model error, so it makes sense to try to 
find the model with minimal error. 
\subsection{Timeline}
\begin{enumerate}
	\item October 18 - Data Simulation
	\item November 8 - Analysis
	\item November 15 - Figures
	\item November 22 - First Draft
	\item December 9 - Projected submission time
\end{enumerate}

\section{Collaboration}
% Provide a description of the way you and 
% your faculty sponsor will collaborate on the project. The faculty sponsor should 
% play a significant role in responding to your ideas, providing advice for new 
% directions and resources, discussing the implications of the results, and 
% helping you prepare for your public presentation. Will there be regularly 
% scheduled meetings between you and your sponsor? Explain how the project relates 
% to the ongoing work of your sponsor, if this is the case.
In order to complete this project, I will be working closely with my faculty advisor, Dr. Brittany Fasy. Additionally, I will be working with Anna Schenfisch, a graduate student in mathematics, and former MSU Post-Doctoral fellow Daniel Salinas. Our group will be meeting once a week on Tuesdays with exceptions of travel. 

%\todo{Does the following apply?  Report on Previous Research Experience (please 
%save and upload this as a 
%separate document): If you have done any previous research as an undergraduate 
%you must include a 1-2 page (double-spaced) summary of your research results or 
%creative products.Please note-if you have received funding from USP or INBRE 
%your proposal will not be considered unless you complete this section.}

% Please draft your proposal in a format that is appropriate for your academic 
% discipline (i.e. MLA, APA, etc.) - consult your mentor if you have questions 
% about what format is most appropriate to your field of study
%%%%%%%%%%%%%%%%%%%%%%%%%%%%%%%%%%%%%%%%%%%%
%% BIBLIOGRAPHY
 \newpage
 \bibliographystyle{acm}
 \begin{thebibliography}{widestlabel}
 \bibitem{pr}
Praggastis, Brenda. Maximal Sections of Sheaves of Data over an Abstract Simplicial Complex. ArXiv:1612.00397v1, 1 Dec. 2016.
\bibitem{mr}
Robinson, Michael. Sheaf and duality methods for analyzing
multi-model systems. arXiv:1604.04647v2 3 Nov. 2016.
\bibitem{intro}
Burnham K.P., Anderson D.R. (1992) Data-Based selection of an Appropriate Biological Model: The Key to Modern Data Analysis. In: McCullough D.R., Barrett R.H. (eds) Wildlife 2001: Populations. Springer, Dordrecht
\bibitem{robinson2017sheaves}
	Robinson, Michael. Sheaves are the canonical data structure for sensor integration. Information Fusion Volume 36 pages 208-224. Elsevier 2017.
\bibitem{bionoise}
Roman Sloutsky, Nicolas Jimenez, S. Joshua Swamidass, Kristen M. Naegle; Accounting for noise when clustering biological data, Briefings in Bioinformatics, Volume 14, Issue 4, 1 July 2013, Pages 423–436, https://doi.org/10.1093/bib/bbs057


 \end{thebibliography}


%%%%%%%%%%%%%%%%%%%%%%%%%%%%%%%%%%%%%%%%%%%%
% References Cited (include in an additional page within the project proposal): 
% Include a list of any literature that you have cited in the proposal. Nearly 
% all 
% good science and engineering proposals cite papers reporting related results, 
% describing the methods to be used or providing background information. Please 
% note-the review panel rarely recommends funding for proposals without adequate
% references.



\end{document}


