\documentclass[12pt]{article}
\date{\today}
\author{Ryan Hansen}
\title{Summary of Previous Research Experience}

\begin{document}
	\maketitle
	\section*{Previous Problem}
	My last experience participating in the Undergraduate Scholar's program was last summer (2018). This was the beginning of the project I am currently proposing to finish next semester. In essence, we noticed it is often difficult to verify various measurements of models, especially in high noise and high dimensional environments. Topology is a fantastic tool to overcome this problem, as it is concerned far more about the shape of data than small variations. Thus, we sought a way to use topology to overcome the aforementioned issue. This culminated in what is known as a sheaf. 
	\section*{Goals}
	There were a number of steps to complete in order to apply this mathematics to a real situation. 
	\subsection{Steps}
	\begin{enumerate}
		\item Define the domain category objects and morphisms
		\item Define the codomain category objects and morphisms
		\item Define how we are mapping the domain to the codomain in a way that preserves the model we are trying to test
		\item Develop an algorithm to tell us if our model is consistent or not, and where it went wrong
		\item Show the algorithm giving correct responses (Consistent and inconsistent data we construct)
		\item Write our findings and method in a cumulative paper.
	\end{enumerate}

	\section*{Experience}
	This section will talk about the actual experience I had over the summer. Since sheaves and topology are still a very active area of research, there was a lot to read into in order to be confident about our approach. Differing papers had differing ideas about how to represent various models as a sheaf, and some of them were far more advantageous than others. This was one of the reasons we decided on modeling stoichiometric equations - they are not a very complicated model. This was the first step - defining the domain category objects and morphisms. The second step posed a challenge, though. After this initial research area, we attempted numerous different sheaf configurations in an attempt to isolate measurable values. Since data is what we are trying to compare against, it is important the sheaf we use gives data in the same manner the data is measured in. Eventually, in the final week of summer, we determined a configuration that would do everything we wanted. This initial research phase clearly took longer than desired and left several goals unfinished, but it was an exceptional learning experience where I got exposure to the real world of research.

	\section*{Accomplishments}
	This section will talk about the research goals I did and did not accomplish, as well as unrelated value I gained. As stated earlier, we only completed the first half of our research goals over the summer. However, I do not think this is necessarily a horrible thing. In our efforts, we did a lot of technical reading, I got exposure to various useful research software (Vim, python packages, git, etc). I also further developed my technical writing and presentation skills. Most importantly, though, we did satisfy the hardest part of the study. From here on out, we have a straight-forward approach of attainable goals. It essentially consists of communicating and coding ideas we already know very well. Since it comprised such a long list of benefits, I definitely want to participate in undergraduate research again.
\end{document}